In 1935, Einstein, Podolsky, and Rosen (EPR) first pointed out the concept of non-local effect in quantum systems in terms of a paradox \cite{EPR_35}. Later in the same year, while trying to explain the paradox, Schr\"{o}dinger \cite{S_35} introduced the term ``entanglement'' and the concept of quantum steering \cite{S_35, WJD_07, JWD_07} which eventually leads to another quantum correlation i.e. Bell-nonlocality \cite{B_64}. From then on, in the literature of quantum information theory, the concept of entanglement as a quantum correlation has consistently been the most dominant one. It is not only of philosophical interest where it deals with deep concepts like realism and locality but also acts as a cardinal resource in various information processing tasks such as teleportation \cite{BBCJPW_93}, super dense coding \cite{BW_92} and many more \cite{BB_84, E_91}. In the recent developments in quantum computation, entanglement is an intricate resource for speed-ups compared to the corresponding classical counterparts \cite{LP_01, JL_03,V_13,GH_23} and also to minimize the effect of environmental noise \cite{DGPM_17}. When multipartite systems are considered, entanglement is one of the most basic forms of non-classical correlation that is also a resource. On the other hand, quantum steering is a stronger form of correlation than entanglement and the steerable states form a strict subset of entangled states. Finally, the Bell non-local states are a strict subset of steerable states. This hierarchy \cite{WJD_07} and application of all these non-local correlations are well explored in literature \cite{TR_11,BCWSW_12}. \\\\
Entangled states are those which are not separable in nature i.e. they can not be written as a convex mixture of product states of the corresponding subsystems. From the perspective of entanglement resource theory \cite{HHH_09}, the local operations and classical communication (LOCC) are the free operations, using which neither entanglement can be created from a separable state, nor can be increased. Hence in this resource theory, the separable states are the free states and they form a convex and compact set.Defining separability of a given mixed state $\rho$, we implicitly assume that the product structure of the composite
Hilbert space is given, $\mathcal{H} = \mathcal{H}_A \otimes \mathcal{H}_B$. This assumption is well justified from the physical point of view. For example,
the EPR scenario distinguishes both subsystems in a natural way. Then we speak about separable (entangled) states, with respect to this particular decomposition of $\mathcal{H}$. Note that any separable pure state may be considered entangled, if analyzed with respect to another decomposition of $\mathcal{H}$.
On the other hand, one may pose a complementary question, interesting merely from the mathematical point of view, which states are separable with respect to any possible decomposition of the $\mathcal{N} = \mathcal{K} \text{ x } \mathcal{M}$ dimensional Hilbert space $\mathcal{H}$. Now it is natural to ask if one has free access to global operations, then is it possible to create entanglement from a separable state? The answer is in general affirmative but is not always true. There is a set of states which is so strongly separable that it is not possible to generate entangled states from them via any global unitary or even via the convex combination of them \cite{KZ_01, VAD_01, KC_01, LHL_03}. These states are called absolute separable (AS) states. Recently, in \cite{PMD_22} the resource theory of non-absolute separability is introduced where global unitaries are the free operations and the non-AS states are the resourceful states. 
%\textcolor{blue}{Absolute local and other references are missing Ref: https://link.springer.com/article/10.1007/s11128-017-1734-4,\\ https://www.worldscientific.com\\/doi/abs/10.1142/S0219749918500405}\\
In this work, we try to connect the effect of indefinite causal order along with post-selection with the possibility of generating some resourceful state starting from AS states using global unitary operations. From the resource theoretic point of view, if we have global unitary operations for free, then can we take states out of the convex set of AS states? We answer the question affirmatively with the help of a quantum switch. Here, we consider that the two global unitary operations represented by two channels $(\mathcal{N}_1, \mathcal{N}_2)$ are acting sequentially. The power of switching is induced by introducing the ancillary system, which dictates that with some probability $\mathcal{N}_2$ acts after $\mathcal{N}_1$ and $\mathcal{N}_1$ acts after $\mathcal{N}_2$ with rest of the probability. Next, to ensure the effect of indefinite causal ordering, we consider the superposition of these two scenarios. Finally a post-selection is done by performing a projective measurement on the ancillary system. The overall effect of the switching action hence consists of the superposition of the global unitaries followed by the post-selection. In the beginning, we consider AS states lying on the boundary of the convex set and show that by using global unitary with switching action, one can take these states out of this set. Next, we consider two-qubit modified Werner state and show how the range of the AS states changes with the value of the parameters in the presence of quantum switch. We notice that it is possible to make even a maximally mixed state resourceful by using the switching operation of suitably chosen unitary. In \cite{LC_10}, the authors identify the geometry of the separable BD states. Here, we extend the results for AS states and show how the structure changes in the presence of switching on global unitary operations. Additionally, we generalize our results numerically by Haar uniformly generating random global unitary matrices. We show our protocol is effective even in higher dimensions.

\section{Motivation}
Quantum switch is a circuit that implements indefinite causal order among a set
of quantum operations (We refer the readers to \cite{OCB_12} for a detailed discussion
on indefinite causal order) \cite{chiribella_2009}. The experimental demonstration of the quantum
switch has been reported in \cite{PMACADHRBW_15},\cite{RRFAZPBW_17}. There exist several applications of the
quantum switch in different information - theoretic/communication tasks. Some of
these applications, we mention below. In \cite{mukhopadhyay_2020},\cite{8966996}, the quantum switch has
been applied in quantum teleportation. In \cite{ESC_18},\cite{PhysRevA.103.062610}, it has been shown that
using quantum switch, the transmission of classical information is possible through
quantum channels that have the maximally mixed state as a fixed output. It has been
shown in \cite{CBBGARSAK_21} that even perfect quantum communication can be achieved using
an entanglement breaking channel using the quantum switch. In \cite{procopio_2019}, quantum
switch has been used on N quantum channels to enhance classical communication.
Quantum switch also has been used to test the properties of quantum channels
\cite{PhysRevA.86.040301},\cite{ACB_14}. In \cite{PhysRevLett.124.190503}, it has been discussed that the quantum switch can be used
to boost the precision of quantum metrology. Quantum switch has been also used
to reduce quantum communication complexity \cite{GFAC_16}. There may be more possible
applications of the quantum switch that are yet to be studied. A part of this thesis
is based on improvement in quantum communication using quantum switch.\\\\
In the recent past, the literature of quantum information theory evidently illustrates that indefinite ordering of multiple causal events can give rise to advantages in terms of resource requirements in different quantum protocols. The concept of indefinite causal ordering was first pointed out in \cite{H_05, H_07} and was extended to resource theoretic frameworks in \cite{CDPV_13}. Exploiting this concept, the structure of quantum switch was introduced \cite{CDPV_13}, where we essentially consider an ancillary system that acts as a controller to the orders of the events. A stronger approach to this indefiniteness is by process matrix formalism introduced in \cite{OCB_12}. Indefinite causal ordering has played a significant role in increasing the efficacy of various information processing activities. These include winning non-local games \cite{OCB_12}, testing properties of quantum channel \cite{C_12}, minimizing quantum communication complexity \cite{GFAC_16}, improving quantum communication \cite{ESC_18, CBBGARSAK_21,Mitra_2023}, increasing the performance of quantum algorithm \cite{ACB_14}, activating non-Markovianity \cite{MB_22,agm_23} and many more. Recent developments in this domain are centered around providing experimental evidence \cite{PMACADHRBW_15, RRFAZPBW_17, GGKCBRW_18}. 

\section{Outline}
The rest of the thesis is organized as follows. In Chapter 2, we discuss the preliminaries that are required for the later parts of the thesis. In Chapter 3, the concepts of Open Quantum Systems and their standard equation are described. Chapter 4 is based on the topic of Indefinite Causal Order and their applications. In Chapter 5, we explain the work done in the paper \cite{yanamandra2023breaking}. We conclude the work and it's results in chapter 6.